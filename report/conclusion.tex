% this file is called up by thesis.tex
% content in this file will be fed into the main document

\chapter{Conclusion} % top level followed by section, subsection

% ----------------------- contents from here ------------------------
% 

This thesis presented the research undertaken to validate the
application of Deep Learning for neutrino detection in the KM3NeT
detector. In light of the results obtained from the empirical
experiments, the research questions are revisited below. Research
questions \textbf{RQ2} and \textbf{RQ3} are answered first followed by
\textbf{RQ1} which was the main research question of the project.

\begin{enumerate}
\item[\textbf{RQ2.}] \textbf{Can the \emph{Hit Correlation Step} be replaced with a Multi Layer Perceptron?}

  The first phase of this project focused on improving the Hit
  Correlation Step of the GPU Pipeline using a Multi Layer Perception.
  In Chapter \ref{cha:mlp} the training and evaluation of such a model
  was presented. The model outperformed the existing Pattern Matrix
  Criterion and was able to identify causally related hits with higher
  accuracy, precision and recall in highly skewed test sets.

\item[\textbf{RQ3.}] \textbf{Can the \emph{Graph Community Detection Step} be replaced with a Graph Convolutional Neural Network?}

  The outcomes of the second phase of this project were described in
  Chapter \ref{cha:gcn}. Here, the Graph Convolutional Neural Network
  to replace the Graph Community Detection and the Classification Step
  of the GPU Pipeline was presented. The performance of the model was
  not ideal since it was biased to the positive class and unable to
  identify the negative class in the test sets.

\item[\textbf{RQ1}.] \textbf{Can the existing GPU pipeline be improved using Neural Networks?}

  The MLP is felt to be a viable successor to the Pattern Matrix
  Criterion due to its superior performance. Although the outcome of
  the GCN model was unfavorable, the report urges that the
  recommendations for its improvements be explored before dismissing
  Graph Neural Networks. Altogether, the project is deemed a success
  as the outcome is sufficient to indicate that Deep Learning can be
  used to improve the performance of the existing GPU Pipeline.
\end{enumerate}


% ---------------------------------------------------------------------------
% ----------------------- end of thesis sub-document ------------------------
% ---------------------------------------------------------------------------

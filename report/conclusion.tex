% this file is called up by thesis.tex
% content in this file will be fed into the main document

\chapter{Conclusion} % top level followed by section, subsection

% ----------------------- contents from here ------------------------
% 

This thesis presented the research undertaken to validate the
application of Deep Learning for neutrino detection in the KM3NeT
detector. In light of the results obtained from the empirical
experiments, the research questions are revisited below. Research
questions \textbf{RQ2} and \textbf{RQ3} are answered first followed by
\textbf{RQ1} which was the main research question of the project.

\begin{enumerate}
\item[\textbf{RQ2.}] \textbf{Can the \emph{Hit Correlation Step} be replaced with a Multi Layer Perceptron?}

  The Hit Correlation Step can be replaced with a Multi Layer
  Perceptron to identify causally related hits. The first phase of
  this project focused on improving the Hit Correlation Step of the
  GPU Pipeline using a Multi Layer Perception. In Chapter
  \ref{cha:mlp} the training and evaluation of such a model was
  presented. The model outperformed the existing Pattern Matrix
  Criterion and was able to identify causally related hits with higher
  accuracy, precision and recall in highly skewed test sets.

\item[\textbf{RQ3.}] \textbf{Can the \emph{Graph Community Detection Step} be replaced with a Graph Convolutional Neural Network?}

  This research question remains open for further exploration due to
  the inconclusive results obtained from experiments conducted in this
  thesis. The outcomes of the second phase of this project were
  described in Chapter \ref{cha:gcn}. Here, the Graph Convolutional
  Neural Network to replace the Graph Community Detection and the
  Classification Step of the GPU Pipeline was presented. The
  performance of the model was not ideal since it was biased to the
  positive class and unable to identify the negative class in the test
  sets.

\item[\textbf{RQ1}.] \textbf{Can the existing GPU pipeline be improved using Neural Networks?}

  The MLP is felt to be a viable successor to the Pattern Matrix
  Criterion due to its superior performance. Although the outcome of
  the GCN model was unfavorable, the report urges that the
  recommendations for its improvements be explored before dismissing
  Graph Neural Networks. Altogether, the project is deemed a success
  as the outcome is sufficient to indicate that Deep Learning can be
  used to improve the performance of the existing GPU Pipeline.
\end{enumerate}

Although the outcomes of this thesis are deemed successful, a few key
challenges faced along the way are worth reflecting upon. Existing
literature recommends the use of precision-recall curves to evaluate
the model performance for skewed datasets. For the KM3NeT dataset
however, this was not the ideal metric due to the severely skewed
distribution of the classes. Most timeslices consist of noise hits in
the order of millions but less than a hundred event hits. This yields
an extremely low precision, even if the model has only a thousand
false positives which is not bad overall.

A rise in the number of false negatives was noticed in the MLP model
with an increase of positive class examples (see Section
\ref{sec:mlp-disc} and Figure \ref{fig:mlp-cm}}). This is a cause for
concern since these are causally related hits which will be filtered
out, resulting in loss of important data for future research. For the
GCN model, a solution to fix the bias was proposed in Section
\ref{sec:rec-gcn} that requires the data to be modeled as a
heterogeneous graph. This can be computationally expensive and
complicates the data preparation procedure. The solution to both
problems stated above is seen in the recommendation proposed in
Section \ref{sec:rec-ind}. The recommendation proposed a single GCN
model to power the entire pipeline using a data preparation
methodology that persists the homogeneous graph structure and renders
the MLP model obsolete altogether.

The \emph{Pytorch Geometric} library was used to construct the graph
neural networks implemented in this thesis. Although this library is
being actively developed and has the support of a healthy open source
community, documentation and support from the maintainers was felt to
be lacking. The \emph{Viltstift} compute cluster provided by Nikhef
was used in this thesis which contained \emph{AMD} GPUs requiring
Pytorch to be built from source. Due to the lack of support for AMD
GPUs in Pytorch Geometric, all experiments pertaining to the GCN model
were conducted on \emph{Google Colab}. Additional restrictions on
memory and disk space posed by Google Colab limited the scale of
experiments to small graphs consisting of no more than 1000 nodes.
Although the small scale experiments sufficiently demonstrated the
promising role of Graph Neural Networks in neutrino detection, the
need for more large scale experiments is observed.

The KM3NeT research initiative seeks to unravel the mysteries of the
Earth and the Universe by studying the elusive neutrino particles. The
Data Acquisition Pipeline (DAP) plays a critical role in this
multi-million euro project since all research efforts hinge upon a
sole component - the data. The data collected through the DAP dictates
the quality and value of all future research thus it must be able to
filter out noise with unparalleled speed and accuracy. This thesis
presented research which can aid in this endeavor using Deep Learning.
It is one of the very few research projects within the KM3NeT sphere
to apply Deep Learning to the field of Particle Physics. It is also
amongst a handful of projects that explore the application of Graph
Neural Networks for neutrino detection. This thesis hopes to have
sufficiently contributed towards the development of the next
generation event trigger algorithms. Being one of the few to apply
Graph Neural Networks for neutrino detection, this thesis hopes to
have laid the foundation for future graph based applications.

% ---------------------------------------------------------------------------
% ----------------------- end of thesis sub-document ------------------------
% ---------------------------------------------------------------------------

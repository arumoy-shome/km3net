% this file is called up by thesis.tex
% content in this file will be fed into the main document

\chapter{Conclusion} % top level followed by section, subsection

% ----------------------- contents from here ------------------------
% 

This report presented the research undertaken to validate the
application of Deep Learning for neutrino detection in the KM3NeT
detector. Chapter \ref{cha:intro} presented the problem space and its
importance, the stakeholders and their requirements and finally the
research questions this project sought to answer. In Chapter
\ref{cha:data}, the procedure for creating the dataset was described.
Statistical and visual analysis of the data revealed the lack of any
correlations or trends amongst the features. Chapter
\ref{cha:related-work} provided more background on the problem by
analyzing the existing literature related to this project. Although
Deep Learning is still an emerging field within the KM3NeT umbrella of
research projects, its presence is felt in the field of Particle
Physics and signal-noise detection. An overview of The GPU Pipeline
upon which this project is based on was provided. Its limitations
were then outlined and motivation for a new pipeline was thus formed.

The first phase of this project focused on improving the Hit
Correlation Step of the GPU Pipeline using a Multi Layer Perception.
In Chapter \ref{cha:mlp} the training and evaluation of such a model
was presented. The model outperformed the existing Pattern Matrix
Criterion and was able to identify causally related hits with higher
accuracy, precision and recall in highly skewed test sets. The
outcomes of the second phase of this project were described in Chapter
\ref{cha:gcn}. Here, the Graph Convolusional Neural Network to replace
the Graph Community Detection and the Classification Step of the GPU
Pipeline was presented. The performance of the model was not ideal
since it was biased to the positive class and unable to identify the
negative class in the test sets.

In Chapter \ref{cha:rec} avenues for future work were discussed.
Recommendations on improving the performance of the MLP, removing the
bias in the GCN and reducing the interdependence of the two models
were made. The MLP is felt to be a viable successor to the Pattern
Matrix Criterion due to its superior performance. Although the outcome
of the GCN model was unfavorable, the report urges that the
recommendations for its improvements be explored before dismissing
Graph Neural Networks. Altogether, the project is deemed a success as
the outcome is sufficient to indicate that Deep Learning can be used
to improve the performance of the existing GPU Pipeline.

% ---------------------------------------------------------------------------
% ----------------------- end of thesis sub-document ------------------------
% ---------------------------------------------------------------------------

Chapter \ref{cha:intro} presented the problem space and its
importance, the stakeholders and their requirements and finally the
research questions this project sought to answer. In Chapter
\ref{cha:data}, the procedure for creating the dataset was described.
Statistical and visual analysis of the data revealed the lack of any
correlations or trends amongst the features. Chapter
\ref{cha:related-work} provided more background on the problem by
analyzing the existing literature related to this project. Although
Deep Learning is still an emerging field within the KM3NeT umbrella of
research projects, its presence is felt in the field of Particle
Physics and signal-noise detection. An overview of The GPU Pipeline
upon which this project is based on was provided. Its limitations
were then outlined and motivation for a new pipeline was thus formed.

In Chapter \ref{cha:rec} avenues for future work were discussed.
Recommendations on improving the performance of the MLP, removing the
bias in the GCN and reducing the interdependence of the two models
were made. 

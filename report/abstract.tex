
% Thesis Abstract -----------------------------------------------------


%\begin{abstractslong}    %uncommenting this line, gives a different abstract heading
\begin{abstracts}        %this creates the heading for the abstract page

  Neutrinos are highly elusive subatomic particles which can only be
  detected with the help of large particle detectors. The KM3NeT
  neutrino telescope is one such detector currently being constructed
  at the bottom on the Mediterranean Sea. Due to its large volume and
  the presence of background noise, ``event trigger'' algorithms are
  utilized by the data acquisition pipeline of the detector to sift
  through the noise. A GPU Pipeline was also developed to improve the
  quality of filtration of the event trigger algorithms without
  compromising their runtime performance. Despite these efforts, the
  quality of filtration require further improvements. The goal of this
  paper is to improve upon the GPU Pipeline using Artificial Neural
  Networks. The paper explores the possibility of replacing parts of
  the GPU Pipeline using Multi Layer Perceptrons and Graph
  Convolusional Neural Networks. The Multi Layer Perception performs
  better compared to the existing solution while the results of the
  Graph Convolusional Network are inconclusive in its existing form.
  Overall, the outcome is promising and new avenues of research are
  discovered through this work.

  \emph{\textbf{Keywords}: Neutrino detection, Artificial Neural
  Network, Multi Layer Perception, Deep learning, Graph Neural
  Networks, Geometric Learning, KM3NeT.}
\end{abstracts}
%\end{abstractlongs}


% ---------------------------------------------------------------------- 

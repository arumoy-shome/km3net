% this file is called up by thesis.tex
% content in this file will be fed into the main document

\chapter{Replacement for Graph Community Detection Step} % top level followed by section, subsection
\label{cha:gcd}
% ----------------------- contents from here ------------------------
% 

This chapter presents the replacement created using a Graph
Convolusional Neural Network (GCN) for the \emph{Graph Community Detection
Step} of the Karas pipeline (see \ref{sec:karas-pipeline}). It is
observed that a GCN is able to classify noise and event nodes very
well, even in extremely skewed datasets with less than 10 event nodes.
The chapter begins with an overview of GCNs and how they have been
applied to this problem. The data preparation, visualization and model
evaluation are touched upon next. The chapter concludes with
discussion of the results and next steps.

\section{Primer on Graph Convolusional Neural Networks}
\label{sec:gcn-primer}

- GCNs operate on graph like data structures
- message passing and aggregation -> node embedding updates
- 
GCNs have several applications however we use it for classification of nodes.
\section{Data Preparation}
\label{sec:gcd-data-prep}

\subsection{Preparation of Training Data}
\label{sec:gcd-data-prep-train}

\subsection{Preparation of Testing Data}
\label{sec:gcd-data-prep-test}

\section{Data Exploration and Visualization}
\label{sec:gcd-data-exp}

\section{Model Description}
\label{sec:gcd-model-desc}

\section{Model Evaluation}
\label{sec:gcd-model-eval}

\section{Discussion}
\label{sec[gcd-disc]}

% ---------------------------------------------------------------------------
% ----------------------- end of thesis sub-document ------------------------
% ---------------------------------------------------------------------------

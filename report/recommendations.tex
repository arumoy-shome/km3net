% this file is called up by thesis.tex
% content in this file will be fed into the main document

\chapter{Recommendations} % top level followed by section, subsection
\label{cha:rec}

% ----------------------- contents from here ------------------------
% 

This chapter presents some practical recommendations for the readers
who wish to use the new data processing pipeline presented in this
report. The chapter also presents alternative paths of research which
remain unexplored and general improvements that can be made to the
pipeline in the future.

Chapter \ref{cha:pm} presented a Multi Layered Perceptron capable of
identifying causally related hits with a higher accuracy, precision
and recall compared to the Pattern Matrix Criterion presented by Karas
et al. The PM model is seem as a viable successor to the PM Criterion.
The model can be further improved using larger training sets and the
model performance can be improved by utilizing the plethora of
techniques.
% TODO what techniques? Name a few relevant ones.

The output of the MLP is observed to be primitive (see Section
\ref{sec:pm-model-disc}). As proved empirically, the GCD model is able
to perform better when the edge weights are assigned based on the type
of connection between the nodes (see Section
\ref{sec:gcd-adv-weights}). In order to obtain the advanced edge
weight scheme, the MLP must be modified such that it performs
multi-class classification on the edge types. For a classification
problem of n edge types, the output of the MLP will thus become a
\texttt{(n,)} vector containing the expected probability for each
class. These probabilities can then be used as the edge weights.
% TODO illustrate this

Chapter \ref{cha:gcd} presented a Graph Convolusional Neural Network
capable of identifying event nodes with immaculate accuracy, precision
and recall in extremely skewed datasets. The network however have a
very high false positive rate when tested with a set containing no
event nodes. This bias may be removed by framing the data as a
multi-edge, heterogeneous graph. The model is thrown off by the high
weights on edges between causally related event nodes and noise nodes.
Thus by creating different types of edges corresponding to the various
types of connections that two nodes may posses (see
\ref{sec:gcd-adv-weights}), each carrying the corresponding weights,
the network may be able to correct it's bias to the positive class.
% TODO illustrate this

The models presented in this report were testing in isolation. However
the intended use is to combine them in order to identify timeslices
containing neutrino event hits (see Section \ref{sec:shomw-pipeline}).
Thus the models should be tested as an integrated pipeline in order to
determine if feasible as improvement over the Karas pipeline.

Results from the GCD experiments (see Section
\ref{sec:gcd-model-disc}) indicate that the model is unable to learn
anything from the node embeddings thus identifying better node
embeddings to help the enhance the model's learning abilities remains
open to be examined. The network currently aggregates the node
embeddings by adding them. This however may not be an apt aggregation
function and the model's performance needs to be compared with
alternatives such as the difference and mean.

An alternative path of research exists to explore the possibility of
replacing the entire pipeline with a single GNN. The data can be
framed such that the \texttt{(x, y, z)} vector is used as the node
embedding and \texttt{$\delta t$} between the nodes is used as the
edge feature.

Finally, the problem can also
be framed as a graph classification problem. The expectation being
that the network is able to identify clusters of causally related
nodes from noise.
% ---------------------------------------------------------------------------
% ----------------------- end of thesis sub-document ------------------------
% ---------------------------------------------------------------------------
